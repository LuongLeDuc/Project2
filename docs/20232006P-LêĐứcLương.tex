\documentclass[a4paper,12pt]{article}

% Các gói cần thiết
\usepackage[utf8]{inputenc}
\usepackage[T5]{fontenc}
\usepackage[utf8]{vietnam}
\usepackage{graphicx}
\usepackage{titlesec}
\usepackage[vietnamese=nohyphenation]{hyphsubst}

% Thiết lập lề
\usepackage[a4paper,top=20mm,bottom=20mm,left=35mm,right=20mm]{geometry}

% Tạo trang bìa
\begin{document}
\begin{titlepage}
  \begin{center}
    \large\textbf{TRƯỜNG ĐẠI HỌC BÁCH KHOA HÀ NỘI}\\
    \textbf{TRUNG TÂM ĐÀO TẠO LIÊN TỤC}\\
    \rule{\textwidth}{0.5pt} \\[5em] % Đường kẻ ngang

    \Huge\textbf{BÁO CÁO}\\[0.5em]
    \LARGE\textbf{PROJECT 1}\\[5em]

    \textbf{\large Đề tài: Xây dựng hệ thống quản lý nhà hàng}\\[4em]

    \textbf{\large Họ tên sinh viên: Lê Đức Lương}\\

    \textbf{\large Lớp: CNTT}\\

    \textbf{\large Giáo viên hướng dẫn:}\\[10em]

    \textbf{\large Hà Nội, tháng \_\_\_ năm \_\_\_}\\
  \end{center}
\end{titlepage}

% Nội dung tài liệu
\section*{\centering \LARGE LỜI NÓI ĐẦU}
Trong thời đại công nghệ số hiện nay, việc áp dụng các giải pháp công nghệ thông tin vào hoạt động kinh doanh ngày càng trở nên phổ biến và thiết yếu. Trong đó, ngành nhà hàng - khách sạn là một lĩnh vực có tiềm năng lớn để cải tiến và nâng cao hiệu quả quản lý thông qua việc ứng dụng các hệ thống thông tin hiện đại.

Dựa trên nhu cầu thực tiễn của các nhà hàng trong việc quản lý đặt bàn, gọi món, lịch làm việc nhân viên, và tối ưu hóa hoạt động kinh doanh, nhóm chúng tôi đã thực hiện đề tài xây dựng Hệ thống quản lý nhà hàng. Đề tài được thực hiện tại [tên đơn vị/tổ chức] trong khoảng thời gian từ [ngày bắt đầu] đến [ngày kết thúc], với mục tiêu:

\begin{enumerate}

  \item Tạo ra một hệ thống quản lý nhà hàng toàn diện, hỗ trợ ba nhóm đối tượng chính là khách hàng, nhân viên, và quản lý.
  \item Nâng cao trải nghiệm khách hàng thông qua các chức năng đặt bàn, gọi món trực tuyến, và tương tác nhanh với nhà hàng.
  \item Tối ưu hóa công việc quản lý nhân sự, lịch làm việc và quy trình xử lý đơn hàng trong nhà hàng.
  \item Cung cấp các công cụ phân tích doanh thu, hiệu suất làm việc và số liệu thống kê chi tiết nhằm hỗ trợ các quyết định kinh doanh chiến lược.
  \item Quá trình thực hiện đề tài không chỉ giúp nhóm phát triển các kỹ năng về thiết kế và triển khai phần mềm mà còn đóng góp vào việc giải quyết các bài toán thực tế trong ngành dịch vụ nhà hàng.
\end{enumerate}

Hy vọng hệ thống sẽ mang lại giá trị thiết thực, đồng thời làm nền tảng cho những cải tiến và ứng dụng công nghệ trong tương lai.

\newpage

\tableofcontents
\newpage

\section{Đặt vấn đề}
\subsection{Giới thiệu}
\subsubsection{Mục đích của hệ thống}
\begin{flushleft}

  Hệ thống quản lý nhà hàng được tạo ra với mục đích giúp người dùng có thể tiếp cận tới các dịch vụ của nhà hàng dễ dàng hơn, mang đến trải nghiệm tốt nhất cho người dùng khi tìm hiểu và sử dụng các dịch vụ của nhà hàng mang lại. Ngoài ra, hệ thống còn giúp nhà hàng quản lý nhân sự, quản lý tài chính và quản lý người dùng, giúp cho nhà hàng hoạt động tốt hơn và giảm tải những công việc thường xuyên lặp đi lặp lại.

\end{flushleft}
\subsubsection{Phạm vi của hệ thống}
\begin{flushleft}
  Hệ thống quản lý nhà hàng 
\end{flushleft}
\subsubsection{Mục tiêu và tiêu chí thành công của hệ thống}
\subsubsection{Định nghĩa và từ viết tắt}
\subsubsection{Tham khảo}
\subsection{Đề xuất hệ thống}
\subsubsection{Tổng quan}
\subsubsection{Yêu cầu chức năng}
\subsubsection{Yêu cầu phi chức năng}
\subsubsection{Mô hình hệ thống}
\subsection{Thuật ngữ}
\section{Cơ sở lý thuyết}
\section{Phân tích hệ thống}
\section{Thiết kế hệ thống}
\section{Thiết kế chương trình}
\section{Kiểm thử chương trình}
\section{Hướng dẫn cài đặt chương trình}
\section{Kết luận}
\subsection{Kết luận}
\subsection{Hướng đi của đồ án trong tương lai}

\end{document}
